\chapter{Introduction}

\par For much of the existence of humankind, we have been enthralled by birds.
Ancient civilizations often used the comings and goings of certain animals as omens of good or bad things to come (CITATION).
These civilizations also figured out how to relate weather patterns to different animals that show up at certain times of year.
Indigenous groups in North America recognized birds as both living creatures and as food sources for hunting, and closely tracked populations of birds within their territories (CITATION).
More recently, humans have formalized the counting of birds.
This started initially with the Christmas Bird Count---which started as an annual hunt and moved toward a census in the early 1900s---where people across North America count all the birds that they can within their Christmas Bird Count circle.
This Christmas Bird Count has provided wintering population numbers for nearly a century (CITE).
In the 1960s, the North American Breeding Bird Survey (BBS herein) began, setting the stage for North America's current gold standard of status and trend estimates for nearly all of North America's Breeding Birds.
Since then, programs such as eBird, iNaturalist, and others have allowed for the proliferation of bird observation data, by enabling all people who enjoy watching birds to submit their sightings.

\par Today, the millions of bird observations available are used across many conservation problems.
The BBS has been used for years in Canada and the US as a metric for deciding when and how to list certain bird species as species-at-risk.
Programs such as the Boreal Avian Modelling project have been able to target remote areas in Canada's Boreal forest to assess the health of the breeding birds there, while also assessing the status of the ecosystem in general.
eBird has been used across numerous conservation projects, from improving our knowledge of secretive marsh birds (TOZER citation), to helping to identify threatened species occurrences outside of their known ranges \citep{lin_using_2022}.
Finally, by combining trend information from the BBS with other targeted surveys, researchers were able to estimate that North America has lost close to three billion birds since the 1970s.
Not only did they find that habitat specialists are declining due to habitat loss, but they also found that common, generalist species such as House Sparrows (\textit{Passer domesticus}) are also declining.

\par While it is important to have these data in conservation decision-making, it is also important to have an understanding of the accuracy, precision, biases, and general limitations of the data being used in these conservation problems.
For most bird surveys, many of these sources of error are accounted for via various modelling approaches.
For example, the BBS accounts for spatial biases and effects by using instrinsic conditional auto-regressive models \citep{besag_bayesian_1991} to explicitly share information among spatial units \citep{smith_patterns_2023}. 
Additionally, the BBS can account for temporal biases and effects, including inter-annual fluctuations, by using generalized additive models \citep{wood_generalized_2017} to flexibly estimate population trajectories \citep{smith_north_2021}.
The BBS also accounts for other nuisance variables such as observer effects, including the effect of whether a particular observer was a first-time observer \citep{kendall_first-time_1996}.
With eBird, because there is less structure to the data that is collected, models of relative abundance often take into account effort hours, peak time of day, and observer skill \citep{johnston_estimates_2018, fink_double_2023}.

\par One metric that is often not accounted for enough in biological surveys is detectability \citep{bennett_how_2024}.
Detectability can be defined as the probability of observing and recording an organism during any biological survey, given the organism is present.
The presence or absence of an animal at a survey location is an important distinction to make so as to separate detectability from occupancy, which is the probability an animal is actually at a given location.
Because of this condition, when assessed as a probability, detectability can be a metric of sensitivity or true positive rate.
As such, detectability becomes important to consider when making conservation decisions based on biological surveys, because it can provide a metric of how confident we are in the survey, particularly with surveys with zeros (GARBAGE FIX).

\par Detectability, and various other metrics derived from detectability, can also be used in other modelling exercises.
For example, detection distance is often used to convert observed abundances in a survey to estimates of true abundance \citep{buckland_introduction_2001, buckland_distance_2015}.
This is one of the foundational assumptions in population estimation, where we try to estimate densities (or true abundance) of organisms based on the number of organisms that we actually observe.
In North America, Partners in Flight maintains a population estimates database that is generated by using data from the BBS and estimates of detection distance based on expert opinion \citep{will_handbook_2020, stanton_estimating_2019}.
These population estimates are of utmost importances to agencies that assess criteria for listing birds based on population sizes, and organizations that manage population numbers (such as for waterfowl and game birds).
Another use of detectability is to correct for systematic biases that occur during the detection process.
For example, the Boreal Avian Modelling project use estimates of detectability to correct for time of year, time of day, and environmental factors during surveys, all of which affect detectability in birds \cite{solymos_lessons_2020, solymos_evaluating_2018, marsh_correcting_1989}.
By applying detectability as a statistical offset, counts from disparate surveys can be corrected for the survey-specific detectability factors, allowing for integration of data.

\par Given the importance of detectability for conservation and management decisions, it is clear to see why we would want to have accurate and precise estimates of detectability for as many organisms as possible.
For many organisms, it is important to be able to even have an estimate to begin with \citep{bennett_how_2024}.
Luckily for the birds, there are millions of data points that exist across hundreds of projects in North America that have been collected over a huge range of spatial, temporal, and environmental scales.
Not only that, but recent advances in modelling have created flexible ways of specifically estimating detectability based on these surveys \citep{solymos_calibrating_2013}, by combining modelling techniques like removal sampling \citep{alldredge_time--detection_2007, farnsworth_removal_2002} and distance sampling \citep{buckland_introduction_2001}.
With the push for more open data in ecology, particularly in this era of Big Data \citep{binley_minimizing_2023}, we are now at a point where we can systematically estimate detectability in landbirds using data-driven approaches, while being able to account for a number of factors that influence detectability.

\par In this thesis, I will focus on the goal of estimating detection probabilities using data-driven approaches.
In particular, I will focus on estimating detectability for landbirds across North America.
This body of research comes at a key moment in bird conservation: we know that populations are decreasing across the continent, and we know that there are systematic biases that prevent us from truly understanding how many birds are actually out there.
Thus, this thesis will seek to use these data-driven approaches to estimates detectability for as many landbird species as possible.
Additionally, I will showcase an application of these data-driven detectability estimates toward the North American Breeding Bird Survey, by applying the detectability estimates as offsets to correct for changes in landscape over time.
Finally, I will provide some thoughts about the future of detectability research, particularly with the advancement of autonomous recording units.

\section{Research Approach}

\par In Chapter 2, entitled ``What affects detectability in landbirds, and how do we model it?", I conduct a literature review that a) describes the detection process of birds in detail, b) reviews the known factors that affect detectability in birds, and c) reviews the methodological advancements and current state of the literature in detectability research.
This chapter will also provide a description of the equations that I will be using throughout the rest of this thesis, particularly equations related to removal modelling and distance modelling.
I will also introduce many of the terms that will be used throughout this thesis.
In essence, this chapter will serve as a review of the known state of detectability research in landbirds, and also serve as a scaffolding chapter for the subsequent data chapters.
By realizing the tools that we have available to us, and by having the biological knowledge of what affects detectability, we can be sure that data-driven estimates of detectability are going to be biologically sound.


\par In Chapter 3, entitled ``Point count offsets for estimating population sizes of North American landbirds", I expand upon the work of \citet{solymos_calibrating_2013} and \citet{solymos_evaluating_2018} by providing a pipeline by which we can systematically estimate detection probabilities for North American landbirds, by accounting for factors of detectability such as time of day, time of year, forest coverage, and road proximity.
Bird monitoring in North America over several decades has generated many open databases, housing millions of structured and semi-structured bird observations. 
These provide the opportunity to estimate bird densities and population sizes, once variation in factors such as underlying field methods, timing, land cover, proximity to roads, and uneven spatial coverage are accounted for. 
To facilitate that integration across databases, I use this chapter to introduce NA-POPS: Point Count Offsets for Population Sizes of North American Landbirds. 
NA-POPS is a large-scale, multi-agency project providing an open-source database of detectability functions for all North American landbirds. 
These detectability functions allow the integration of data from across disparate survey methods using the QPAD approach, which considers the probability of detection (q) and availability (p) of birds in relation to area (a) and density (d). 
As of this chapter, NA-POPS has compiled over 7.1 million data points spanning 292 projects from across North America, and produced detectability functions for 338 landbird species. 
In this chapter, I describe the methods used to curate these data and generate these detectability functions, as well as the open-access nature of the resulting database. 
I also provide a worked example of estimating probabilities of detection across a range of covariates for American Robin (\textit{Turdus migratorius}).


\par The broad-scale estimation of detection probabilities works well for species where we have sufficient data, but fails to provide estimates for species of landbirds that are rare or undersampled.
There is therefore a need for methodology that provides detection estimates for species that are rare or undersampled, which generally lack sufficient data for accurate estimates. 
Thus, in Chapter 4, entitled ``Predicting detection probabilities for rare and undersampled North American landbirds", I take advantage of similarities in detection probabilities among phylogenetically-related species and species with similar traits to estimate detection probabilities in seven species of rare North American birds.
I develop a hierarchical Bayesian multi-species removal model and distance model to facilitate this modelling approach, which allows species with fewer data points to ``borrow" information from similar species with more data points via statistical shrinkage. 
I show that a multi-species model improves precision of detectability estimates for species with low sample size, and even with species up to 1000 observations.
However, when no data were available for a species, the multi-species models tended to be overly conservative in predicting detectability. 

\par While Chapters 3 and 4 focus on the data-driven estimates of detectability, Chapter 5, entitled ``A point-level trend model for the North American Breeding Bird Survey that corrects for detectability", focuses on an application of these estimates.
Here, I modify the current status and trend model for the BBS \citep{smith_spatially_2023} to allow it to explicitly account for detectability in trend calculation, while also propagating forward the variance associated with the estimated detection functions.
I demonstrate, using a case study of a forest-dwelling warbler in Canada, how not accounting for detectability in the BBS can given us very different estimates of trends on fine-scales, because we otherwise cannot account for changes in landscape that affect changes in detectability.

\par Chapter 6, entitled ``Conservation Implications: Why should we care about detectability?" takes a turn away from the mathematical models of Chapters 2 through 5, and examines the implications that all this modelling has for conservation science as a field.
In particular, I focus on several of the use cases of systematically-derived detectability estimates, and why we should care about having accurate and precise estimates.
This chapter was inspired by a plenary session given by Professor Gerardo Ceballos at the 2022 Ecological Society of American/Canadian Society for Ecology and Evolution joint conference in Montreal, Quebec, Canada.
In the last 15 minutes of his plenary, he discussed his advice for upcoming and current graduate students in the field of ecology, evolution, conservation science, and other related fields.
His one piece of advice (to paraphrase him) was to ``make sure that every single piece of research you do has a conservation implication to it", and to ``explicitly make a `conservation implications' section of every single paper that you write".
Statistical ecology (a field which I believe this thesis firmly falls into) can often be an esoteric field, with conservation implications only being realized far after methods have been developed.
Therefore, my goal with this chapter is to explicitly discuss what I feel are important implications for conservation that stems from this work, with the hope that this will continue to spur discussions related to detectability research and to inspire ideas about how to model detectability, how to apply detectability, etc.

\par Finally, in Chapter 7, I look toward the future of detectability research and pose (TO DO, INSERT NUMBER) research questions and avenues that I feel are the most important questions in detectability research moving forward.
These questions stem from multiple fronts, ranging from the research output of Chapters 3 - 5, to conversations with colleagues in a detectability working group, to conversations with biologists in the field.
I also provide a small case study of looking toward alternative ways of estimating detectability, specifically using autonomous recording units.
Up until recent decades, most bird data have been collected by humans in the field.
However, humans are notoriously bad at judging distance to birds, which often results in poor estimates of detection distance.
In the case study, I suggest a pipeline by which researchers can estimate detectability using autonomous recording units, and provide my thoughts on the several directions that this body of research could go.

